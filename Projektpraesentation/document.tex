% !TeX TS-program = xelatex

\documentclass{beamer}
\usepackage{xltxtra}
\usepackage{fontspec}
\usepackage{polyglossia}
\usepackage{makecell}
\setdefaultlanguage{german}

\usetheme[blue]{Verona}
\usefonttheme{default}
\usefonttheme{professionalfonts}

\usefonttheme[stillsansseriftext]{serif}

\defaultfontfeatures{Ligatures=TeX,Scale=MatchLowercase}
\setsansfont{Segoe UI Light}
\setmainfont{Segoe UI}
\setmonofont{Impact}




\title{Message Queuing Telemetry Transport}
\subtitle{Implementierung einer IoT-Anwendung auf Basis von MQTT}
\author{Maximilian Gaul, Lukas Dorner}
\date{1.07.2019}

%\mode<presentation>

\begin{document}
	
\begin{frame}
	\titlepage
\end{frame}

\begin{frame}
\frametitle{MQTT}
\framesubtitle{Control Packet}
Besteht aus bis zu drei Teilen:
\begin{itemize}
	\item \textbf{Fixed Header} - in allen MQTT Paketen vorhanden
	\item \textbf{Variable Header}
	\item \textbf{Payload}
\end{itemize}

\end{frame}

\begin{frame}
\frametitle{MQTT}
\framesubtitle{Fixed Header}
\begin{tabular}{c|c|c|c|c|c|c|c|c}
	\hline
	\textbf{Bit $\rightarrow$} & 7 & 6 & 5 & 4 & 3 & 2 & 1 & 0\\
	\hline
	Byte 1 & \multicolumn{4}{|c|}{MQTT Control Packet type} & \multicolumn{4}{c}{Flags specific to each MQTT Control Packet type}\\
	\hline
	Byte 2 & \multicolumn{8}{c}{Remaining Length}\\
	\hline
\end{tabular}

\end{frame}

\begin{frame}
\frametitle{MQTT}
\framesubtitle{Fixed Header - Byte 1}
\begin{itemize}
	\item \textit{Control Packet Type}[7:4] gibt an, welche Art von Paket versendet wird:
	\begin{itemize}
		\item \begin{tabular}{ll}
			\textit{CONNECT} & Client will sich mit dem Server verbinden
		\end{tabular}
		\item \begin{tabular}{ll}
			\textit{CONNACK} & Verbindungs-\textit{ACK}
		\end{tabular}
		\item \begin{tabular}{ll}
			\textit{PUBLISH} & \hspace*{1em}\makecell[l]{Sensor schickt neuen Wert an Server}
		\end{tabular}
		\item \begin{tabular}{ll}
			\textit{SUBSCRIBE} & \makecell[l]{Client \textit{abonniert} ein Thema,\\Server leitet \textit{PUBLISH} weiter an Abonnenten}
		\end{tabular}
	\end{itemize}
	\item \textit{Flag}[3:0] sind spezifisch je nach Control Packet Type gesetzt\\
	ungültige Flags führen zu einem Schließen der Verbindung durch den Empfänger
	\begin{itemize}
		\item \begin{tabular}{ll}
			\textit{DUP} & \makecell[l]{$0$ := Erster Versuch, ein \textit{PUBLISH} zu senden,\\$1$ := Möglicherweise erneutes Senden eines \textit{PUBLISH}}
		\end{tabular}
		\item \begin{tabular}{ll}
			\textit{QoS} & \makecell[l]{Gibt an, wie oft ein \textit{PUBLISH} maximal bzw. minimal gesendet wird}
		\end{tabular}
			\item \begin{tabular}{ll}
		\textit{RETAIN} & \makecell[l]{$1$ := Server \textit{muss} Nachricht speichern \& an \textit{zukünftige}\\\hspace*{2em}Abonnenten senden\\$0$ := Server \textit{darf nicht} $1$}
	\end{tabular}
	\end{itemize}
\end{itemize}

\end{frame}

\end{document}